\begin{abstract}
El objetivo de este documento es describir el problema presentado, y las diversas soluciones que se fueron realizando utilizando distintas tecnicas de programaci\'on, para la solucion exacta, o diversas heur\'isticas para resolver el problema de forma polinomial, dado que no se conoce algoritmo de  solucion exacta con complejidad polinomial.\\

El problema que se nos presenta es, dado un grafo $G=(V, E)$, dos v\'ertices $u,v \in V$, dos funciones $w_1$ y $w_2$ de costo asociadas a las aristas del grafo, y un valor $K \in \mathbb{R}$, se pide resolver el problema de Camino Acotado de Costo Minimo (CACM), el cual consiste en encontrar un camino P entre $u, v$ tal que el costo del camino respecto a la funcion $w_2$ sea minimo entre todos los caminos con $w_1 \leq K$ y valga la condicion del costo del camino $w_1 \leq K$.\\

Se nos requiri\'o modelar situaciones de la vida real con CACM y luego implementar diversas soluciones para este problema, preferimos C y C++ como lenguajes de programaci\'on de este trabajo practico y en el fueron implementados los siguientes algoritmos:
\begin{itemize}
	\item Soluci\'on exacta (C/C++)
	\item Heur\'istica constructiva golosa (C++)
	\item Heur\'istica de busqueda local (C++)
	\item Metaheur\'istica GRASP (C++)
\end{itemize}

Todas las soluciones listadas arriba fueron testeadas con diversos casos de testing y ser\'an claramente documentadas en las secciones de este documento.
Para cada punto se detalla el algoritmo, se realizan justificaciones acerca de su correcto funcionamiento donde sea necesario, se establece una cota teorica sobre la complejidad temporal, y se brindan ejemplos de funcionamiento.
Ademas se realizaron benchmarks de performance sobre diferentes conjuntos de instancias distintas de entrada que afirman la complejidad teorica y son presentadas con graficos que acompan\~an dichos experimentos.
Para las heur\'isticas golosas y de busqueda local se detallan familias o instancias malas de entradas donde la solucion provista no es la optima, se realizaron diversos analisis de optimalidad y variacion de parametros de las heuristicas detallados en cada seccion de cada heuristica.

\vspace{0.5cm}

Para la metaheuristica \texttt{GRASP} se ajustaron los parametros para obtener cierto tradeoff, y fueron fijadas en esos parametros para la seccion de experimentacion general.

\vspace{0.5cm}

Finalmente se realiz\'o una experimentacion general con respecto al rendimiento de las heuristicas respecto a exacta, las heuristicas solas entre ellas para entradas mas grandes de las que soporta el algoritmo exacto en la practica, los experimentos miden y comparan la cercania de las soluciones respecto a la solucion optima con diversos estimadores estadisticos, asi como tambien el tiempo insumido en obtener las soluciones para realizar un tradeoff entre porcentaje de efectividad de la heuristica y tiempo consumido por el algoritmo.
\end{abstract}

