\section{Heuristicas de busqueda local}
\subsection{Explicacion detallada del algoritmo propuesto}
Sea $s$ una solucion factible, es decir, un camino entre $u,v$ tal que el costo del camino $w_1 \leq K$,
para plantear esta heur\'istica definimos $N(s) = \{ $ conjunto de soluciones vecinas de s$\}$ = $\{$ caminos entre u,v tal que $w_1 \leq K$ y difieren en solo un nodo de $s\}$.\\

Se plantearon 3 posibles enfoques para aplicar busqueda local sobre esta vecindad, a continuacion se explicar\'an cada uno.
\subsubsection{Obtencion de la solucion inicial factible}

Para obtener la solucion inicial factible de nuestra heur\'istica ejecutamos el algoritmo de camino m\'inimo de Dijkstra, entre los nodos u y v, sobre la funcion $w_1$, acto seguido validamos si $dist(w_1, src, dst) > K$ entonces no existe soluci\'on factible, caso contrario tenemos una soluci\'on factible inicial para comenzar las 
iteraciones de busqueda local.

\subsubsection{Criterio de terminaci\'on}
El criterio de terminacion fue repetir las iteraciones sobre las nuevas soluciones que se iban obteniendo, hasta que no se obtiene mejora.
Nota: siempre se aplica uno solo de los 3 metodos a las iteraciones, es decir, no se combinan.

\subsubsection{Reemplazar un nodo intermedio en una 3-upla consecutiva de nodos del camino}
Si el camino tiene longuitud 2, es decir hay una arista directa entre $u,v$ , no hay nada que hacer en este caso, caso contrario, sea $S = \{u,...,v_l, v_{l+1}, v_{l+2}, v_{l+3} ..., v\}$ la solucion actual. Lo que hace en este caso es iterar sobre todas las 3-uplas consecutivas del camino, en este ejemplo sea $(vl, v_{l+1}, v_{l+2})$ una 3-upla, $(v_{l+1}, v_{l+2}, v_{l+3})$ la siguiente a iterar, etc...\\ \\
Para cada 3-upla iterada se revisan los vecinos en comun entre los extremos de la 3-upla buscando una mejor conexi\'on entre extremos reemplazando el nodo intermedio 
como se indica en el grafico debajo, es decir, se busca una nueva conexion entre los extremos tal que, si el nuevo nodo intermedio es $v_t$, vale que:
\begin{itemize}
	\item Mejore la conexion de la 3-upla respecto a $w_2$, $w_2(vl, v_{t}) + w_2(vt, v_{l+2}) < w_2(vl, v_{l+1}) + w_2(v_{l+1}, v_{l+2})$
	\item Este cambio, reflejado en la nueva solucion candidata S', no sobrepase la cota K establecida sobre $w_1$, es decir, sea factible.
\end{itemize}


	INSERTAR GRAFICO ACA!


Una iteracion consiste en recorrer todas las 3-uplas del camino obteniendo de los vecinos en com\'un de los extremos de cada 3-upla, si es posible, la mejor forma de mejorar esta conexion, ademas recordando la mejor 3-upla para aplicar la mejora a lo largo de la iteraci\'on, tal que el camino resultante de aplicar esta mejora sea el minimo sobre $w_2$ en la vecindad $N(s)$. Mas formalmente, se busca una solucion vecina del camino S, tal que siendo $w_1(P), w_2(P)$ los costos sobre $w_1$ y $w_2$ respectivamente de una solucion, y $S' = S \setminus \{(v_k, v_{k+1}); (v_{k+1}, v_{k+2})\} \cup \{(v_k, v_t);(v_t, v_{k+2})\}$ la nueva solucion obtenida, entonces valga que:
\begin{itemize}
	\item Sea factible, pertenezca a $N(S)$, $ w_1(S') \leq K$
	\item Sea minima respecto de $w_2$ en la vecindad $N(S)$, $w_2(S') \leq w_2(T)$ $ \forall$ $ T \in N(S) $
\end{itemize}


\subsubsection{Insertar un nodo intermedio en un par consecutivo de nodos del camino}
Este enfoque es basicamente igual al anterior, solo que en lugar de iterar sobre 3-uplas reemplazando el nodo intermedio, se itera sobre pares consecutivos de nodos de la solucion S $(v_k, v_{k+1})$ buscando si existe, un vecino en comun entre los extremos tal que el sendero $(v_k, v_t, v_{k+1})$ tenga costo menor(de forma minima entre los vecinos) sobre $w_2$ que la arista directa y que reflejado en la solucion candidata S' que surge de aplicar este cambio, siga siengo factible($w_2(S') \leq K$). La iteracion de sigue siendo sobre toda la vecindad $N(S)$ y quedandose con el mejor $S' \in N(S)$ posible.

\subsection{Complejidad temporal para el peor caso}
\subsection{Nivel de optimalidad de las soluciones}% En grasp no hace falta este inciso por enunciado.
%Describir si es posible instancias de CACM para las cuales el metodo no proporciona una solucion optima.
%Indicar si es posible que tan mala puede ser la solucion obtenida respecto a la solucion optima
%\subsection{Realizar una expermientacion blablabla...todo lo que respecta a testing y benchmarking tiene una subsection para cada punto del tp en la seccion de benchmarking y por ende no hace falta hablar de eso aca}
\subsection{Casos bien y casos mal}
\subsection{Taboo list caminos del ciclo}
\subsection{Porque elegir la mejor 3-upla de cada it y no la primera, pensar en siguientes iteraciones que podrian ser mejor}